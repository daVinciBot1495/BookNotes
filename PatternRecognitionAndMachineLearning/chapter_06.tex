\documentclass[12pt]{article}

\usepackage{amsmath}

\newcommand{\x}{\mathbf{x}}
\newcommand{\w}{\mathbf{w}}
\newcommand{\tVec}{\mathbf{t}}
\newcommand{\aVec}{\mathbf{a}}
\newcommand{\sumN}{\sum \limits_{n=1}^N}
\newcommand{\Ja}{J\left(\aVec{}\right)}
\newcommand{\Jw}{J\left(\w{}\right)}
\newcommand{\wTransW}{\w{}^\top\w{}}
\newcommand{\tTransT}{\tVec{}^\top\tVec{}}
\newcommand{\dJdW}{\frac{\partial \Jw{}}{\partial \w{}}}
\newcommand{\dJda}{\frac{\partial \Ja{}}{\partial \aVec{}}}
\newcommand{\phiX}{\mathbf{\phi}\left(\x{}\right)}
\newcommand{\phiXOne}{\mathbf{\phi}\left(\x{}_1\right)}
\newcommand{\phiXn}{\mathbf{\phi}\left(\x{}_n\right)}
\newcommand{\phiXm}{\mathbf{\phi}\left(\x{}_m\right)}
\newcommand{\wTransPhi}{\w{}^\top\phiXn{}}
\newcommand{\phiTransW}{\phiXn{}^\top\w{}}
\newcommand{\PhiMat}{\mathbf{\Phi}}
\newcommand{\PhiPhiTrans}{\PhiMat\PhiMat^\top}
\newcommand{\K}{\mathbf{K}}
\newcommand{\I}{\mathbf{I}}

\begin{document}
\section{Kernel Methods}
\subsection{Dual Representations}
Here we consider a linear regression model
%
\begin{equation*}
  y\left(\x{}\right) = \w{}^\top\phiX{}
\end{equation*}
%
whose parameters are determined by minimizing a regularlized sum-of-squares
error function
%
\begin{equation*}
  \Jw{} = \frac{1}{2}\sumN{}\left[\wTransPhi{} - t_n\right]^2 +
  \frac{\lambda}{2}\wTransW{}
\end{equation*}
%
Take the derivative of $\Jw{}$ with respect to $\w{}$, set it equal to $0$, and
solve for $\w{}$. The first step is to expand the square
%
\begin{align*}
  \Jw{} &= \frac{1}{2}\sumN{}
  \left[\left(\wTransPhi{} - t_n\right)\left(\wTransPhi{} - t_n\right)\right] +
  \frac{\lambda}{2}\wTransW{} \\
  \Jw{} &= \frac{1}{2}\sumN{}
  \left[\wTransPhi{}\wTransPhi{} - 2t_n\wTransPhi{} + t_n^2\right] +
  \frac{\lambda}{2}\wTransW{}
\end{align*}
%
Now, note that $\wTransPhi{}$ is a scalar, and can be rewritten as
$\phiTransW{}$. Substituting into the equation above
%
\begin{equation*}
  \Jw{} = \frac{1}{2}\sumN{}
  \left[\wTransPhi{}\phiTransW{} - 2t_n\wTransPhi{} + t_n^2\right] +
  \frac{\lambda}{2}\wTransW{}
\end{equation*}
%
Taking the derivative of $\Jw{}$ with respect to $\w{}$
%
\begin{align*}
  \dJdW{} &= \frac{1}{2}\sumN{}
  \left[2\phiXn{}\phiXn{}^\top\w{} - 2t_n\phiXn{}\right] + \lambda\w{} \\
  \dJdW{} &= \sumN{}\phiXn{}\left[\phiTransW{} - t_n\right] + \lambda\w{} \\
  \dJdW{} &= \sumN{}\left[\wTransPhi{} - t_n\right]\phiXn{} + \lambda\w{}
\end{align*}
%
Setting the above equation equal to $0$ and solving for $\w{}$
\begin{align*}
  0 &= \sumN{}\left[\wTransPhi{} - t_n\right]\phiXn{} + \lambda\w{} \\
  \w{} &= -\frac{1}{\lambda}\sumN{}\left[\wTransPhi{} - t_n\right]\phiXn{}
\end{align*}
%
Now, let $a_n = -\frac{1}{\lambda}\left[\wTransPhi{} - t_n\right]$ and
$\PhiMat{}$
be the design matrix
%
\begin{equation*}
  \PhiMat{} =
  \left[\begin{matrix}
    \phiXOne^\top\\
    \vdots\\
    \phiXn^\top
  \end{matrix}\right]
\end{equation*}
%
Substituting $a_n$ into the equation for $\w{}$
%
\begin{equation*}
  \w{} = \sumN{}a_n\phiXn{}
\end{equation*}
%
Hence, $\w{}$ is a linear combination of of the columns of $\PhiMat{}^\top$.
This may be written in vector/matrix notation as
%
\begin{equation*}
  \w{} = \PhiMat{}^\top\aVec{}
\end{equation*}
%
where $\aVec{} = \left[a_1 \hdots a_n\right]^\top$.

Before substituting the equation for $\w{}$ into $\Jw{}$, let's rewrite it in
vector/matrix notation
%
\begin{align*}
  \Jw{} &= \frac{1}{2}\sumN{}
  \left[\wTransPhi{}\phiTransW{} - 2t_n\wTransPhi{} + t_n^2\right] +
  \frac{\lambda}{2}\wTransW{}\\
  \Jw{} &= \frac{1}{2}\sumN{}\wTransPhi{}\phiTransW{} -
  \frac{1}{2}\sumN{}2t_n\wTransPhi{} +
  \frac{1}{2}\sumN{}t_n^2 +
  \frac{\lambda}{2}\wTransW{}\\
  \Jw{} &= \frac{1}{2}\w{}^\top\sumN{}\left[\phiXn{}\phiXn^\top\right]\w{} -
  \w{}^\top\sumN{}t_n\phiXn{} +
  \frac{1}{2}\sumN{}t_n^2 +
  \frac{\lambda}{2}\wTransW{}\\
  \Jw{} &= \frac{1}{2}\w{}^\top\PhiMat{}^\top\PhiMat{}\w{} -
  \w{}^\top\PhiMat{}^\top\tVec{} +
  \frac{1}{2}\tTransT{} +
  \frac{\lambda}{2}\wTransW{}
\end{align*}
%
where $\tVec{} = \left[t_1 \hdots t_n\right]^\top$. Now, we may substitue the
equation for $\w{}$ into $\Jw{}$ to yield
%
\begin{equation*}
\Ja{} = \frac{1}{2}\aVec{}^\top\PhiPhiTrans{}\PhiPhiTrans{}\aVec{} -
  \aVec{}^\top\PhiPhiTrans{}\tVec{} +
  \frac{1}{2}\tTransT{} +
  \frac{\lambda}{2}\aVec{}^\top\PhiPhiTrans{}\aVec{}
\end{equation*}
%
To simplify, let $\K{} = \PhiPhiTrans{}$ be the Gram matrix, where
%
\begin{equation*}
\K{}_{nm} = \phiXn{}^\top\phiXm{} = k\left(\x{}_n,\x{}_m\right)
\end{equation*}
%
Hence, $\K{}$ is all pairwise evaluations of the kernel function over the
training set. Plugging this into $\Ja{}$ results in
%
\begin{equation*}
  \Ja{} = \frac{1}{2}\aVec{}^\top\K{}\K{}\aVec{} -
  \aVec{}^\top\K{}\tVec{} +
  \frac{1}{2}\tTransT{} +
  \frac{\lambda}{2}\aVec{}^\top\K{}\aVec{}
\end{equation*}
%
Taking the derivative of $\Ja{}$ with respect to $\aVec{}$
\begin{equation*}
  \dJda = \K{}\K{}\aVec{} - \K{}\tVec{} + \lambda\K{}\aVec{}
\end{equation*}
%
Setting this equal to $0$ and solving for $\aVec{}$
\begin{align*}
  0 &= \K{}\K{}\aVec{} - \K{}\tVec{} + \lambda\K{}\aVec{}\\
  0 &= \K{}\K{}\aVec{} + \lambda\K{}\aVec{} - \K{}\tVec{}\\
  0 &= \K{}\K{}\aVec{} + \lambda\K{}\I{}\aVec{} - \K{}\tVec{}\\
  0 &= \K{}\left(\K{}\aVec{} + \lambda\I{}\aVec{}\right) - \K{}\tVec{}\\
  0 &= \K{}\left(\K{} + \lambda\I{}\right)\aVec{} - \K{}\tVec{}\\
  \K{}\tVec{} &= \K{}\left(\K{} + \lambda\I{}\right)\aVec{}\\
  \tVec{} &= \left(\K{} + \lambda\I{}\right)\aVec{}\\
  \aVec{} &= \left(\K{} + \lambda\I{}\right)^{-1}\tVec{}
\end{align*}
%
Finally, we may rewrite our linear regression model as
%
\begin{align*}
  y\left(\x{}\right) &= \w{}^\top\phiX{}\\
  y\left(\x{}\right) &= \phiX{}^\top\w{}\\
  y\left(\x{}\right) &= \phiX{}^\top\PhiMat{}^\top\aVec{}\\
  y\left(\x{}\right) &= \mathbf{k}\left(\x{}\right)^\top\aVec{}\\
  y\left(\x{}\right) &= \mathbf{k}\left(\x{}\right)^\top\left(\K{} + \lambda\I{}\right)^{-1}\tVec{}\\
\end{align*}
%
where
%
\begin{equation*}
  \mathbf{k}\left(\x{}\right) = 
  \left[\begin{matrix}
    k\left(\x{}_1, \x{}\right)\\
    \vdots\\
    k\left(\x{}_n, \x{}\right)
  \end{matrix}\right]
\end{equation*}
%
Thus we see that the dual formulation allows the solution to the least-squares
problem to be expressed entirely in terms of the kernel function
$k\left(\x{}, \x{}^\prime\right)$. This means that the feature vector $\phiX{}$
never has to be explicitly computed.
\end{document}
