\documentclass[12pt]{article}

\usepackage{amsmath}

\newcommand{\lp}{\left(}
\newcommand{\rp}{\right)}
\newcommand{\lb}{\left[}
\newcommand{\rb}{\right]}
\newcommand{\x}{\mathbf{x}}
\newcommand{\w}{\mathbf{w}}
\newcommand{\aVec}{\mathbf{a}}
\newcommand{\yOfX}{y\left(\x\right)}
\newcommand{\yOfXn}{y\left(\x_n\right)}
\newcommand{\sumN}{\sum \limits_{n=1}^N}
\newcommand{\phiOfX}{\mathbf{\phi}\left(\x\right)}
\newcommand{\phiOfXn}{\mathbf{\phi}\left(\x_n\right)}
\newcommand{\wTPhiOfXn}{\w^\top\phiOfXn}
\newcommand{\phiOfXnTw}{\phiOfXn^\top\w}
\newcommand{\LFunc}{L\lp\w,b,\aVec\rp}
\newcommand{\dLdw}{\frac{\partial \LFunc}{\partial \w}}
\newcommand{\dLdb}{\frac{\partial \LFunc}{\partial b}}
\DeclareMathOperator*{\argmax}{arg\,max}
\DeclareMathOperator*{\argmin}{arg\,min}

\begin{document}
\section{Sparse Kernel Machines}
\subsection{Maximum Margin Classifiers}

We begin our discussion of support vector machines by returning to the two-class
classification problem using linear models of the form
%
\begin{equation*}
  \yOfX = \w^\top\phiOfX + b
\end{equation*}
%
where $\phiOfX$ denotes a fixed feature-space transformation, and we have made
the bias parameter $b$ explicit. The training data set comprises $N$ input
vectors $\x_1,\hdots,\x_N$, with corresponding target values $t_1,\hdots,t_N$
where $t_n \in \{-1, 1\}$, and new data points $\x$ are classified
according to the sign of $\yOfX$.

We shall assume for the moment that the training data set is linearly separable
in the feature space, so that by definition there exists at least one choice of
the parameters $\w$ and $b$ such that $\yOfXn > 0$ for points having $t_n = +1$
and $\yOfXn < 0$ for points having $t_n = -1$, so that $t_n\yOfXn > 0$ for all
training data points.

Recall that the perpendicular distance of a point $\x$ from a hyperplane defined
by $\yOfX = 0$ is given by $|\yOfX|/\|\w\|$. Furthermore, we are only interested
in solutions for which all data points are correctly classified, so that
$t_n\yOfXn > 0$ for all $n$. Thus the distance of a point $\x_n$ to the decision
surface is given by
%
\begin{equation*}
  d_n =
  \frac{|\yOfXn|}{\|\w\|} =
  \frac{t_n\yOfXn}{\|\w\|} =
  \frac{t_n\lp\wTPhiOfXn + b\rp}{\|\w\|}
\end{equation*}
%
The margin is given by the perpendicular distance of the point $\x_n$ that is
closest to the decision surface, and we wish to determine the $\w$ and $b$ that
maximize the margin. Thus the maximum margin solution is found by solving
%
\begin{align*}
  MMS &= \underset{\w,b}{\argmax}\left\{\underset{n}{\min}\lb\frac{t_n\lp\wTPhiOfXn + b\rp}{\|\w\|}\rb\right\} \\
  MMS &= \underset{\w,b}{\argmax}\left\{\frac{1}{\|\w\|}\underset{n}{\min}\lb t_n\lp\wTPhiOfXn + b\rp\rb\right\}
\end{align*}
%
where we have taken the factor $1 / \|\w\|$ outside the optimization over $n$
because $\w$ does not depend on $n$. We can simplify this optimization problem
by scaling $\w$ and $b$ by a constant $\kappa$ such that
%
\begin{equation*}
  t_n\lp\wTPhiOfXn + b\rp = 1
\end{equation*}
%
for the point that is closest to the decision surface. This scaling is
permissible because it does not change the distance from any point $\x_n$ to the
decision surface
\begin{align*}
  d_n &= \frac{t_n\lb\lp\kappa\w\rp^\top\phiOfXn + \kappa b\rb}{\|\kappa\w\|} \\
  d_n &= \frac{t_n\lb\kappa\wTPhiOfXn + \kappa b\rb}{\|\kappa\w\|} \\
  d_n &= \frac{t_n\lb\kappa\wTPhiOfXn + \kappa b\rb}{\sqrt{\lp\kappa\w\rp^\top\lp\kappa\w\rp}} \\
  d_n &= \frac{t_n\lb\kappa\wTPhiOfXn + \kappa b\rb}{\sqrt{\kappa^2\w^\top\w}} \\
  d_n &= \frac{\kappa t_n\lb\wTPhiOfXn + b\rb}{\kappa\sqrt{\w^\top\w}} \\
  d_n &= \frac{t_n\lb\wTPhiOfXn + b\rb}{\|\w\|}
\end{align*}
%
Now, after scaling $\w$ and $b$ by $\kappa$
%
\begin{equation*}
  \underset{n}{\min}\lb t_n\lp\wTPhiOfXn + b\rp\rb = 1
\end{equation*}
%
and
%
\begin{equation*}
  t_n\lp\wTPhiOfXn + b\rp \ge 1
\end{equation*}
%
for $n = 1,\hdots,N$. This reduces the maximum margin solution to
%
\begin{equation*}
  MMS = \underset{\w,b}{\argmax}\frac{1}{\|\w\|}
\end{equation*}
%
or equivalently
%
\begin{equation*}
  MMS = \underset{\w,b}{\argmin}\frac{1}{2}\|\w\|^2
\end{equation*}

This is an example of a quadratic programming problem in which we are trying to
minimize a quadratic function subject to a set of linear inequality constraints.

Solving this constrained optimization problem can be simplified by embedding the
constraints into the objective function being minimized. First, let's rewrite
the linear inequality constraints as
%
\begin{equation*}
  t_n\lp\wTPhiOfXn + b\rp - 1 \ge 0
\end{equation*}
%
Note that for selections of $\w$ and $b$ that incorrectly classify training
examples the left hand side of the above equation will be negative, which
violates the constraint. However for correctly classified training examples
the left hand side will be positive. Hence, we want to maximize
%
\begin{equation*}
  \sumN a_n \lb t_n\lp\wTPhiOfXn + b\rp - 1 \rb
\end{equation*}
%
where we have introduced Lagrange multipliers $a_n \ge 0$, with one multiplier
$a_n$ for each of the linear inequality constraints. Combining this with the
maximum margin objective function results in the Langrangian function
%
\begin{equation*}
  \LFunc = \frac{1}{2}\|\w\|^2 - \sumN a_n \lb t_n\lp\wTPhiOfXn + b\rp - 1 \rb
\end{equation*}
%
where $\aVec = \lb a_1 \hdots a_N \rb^\top$. Note the minus sign in front of the
Lagrange multiplier term, because we are minimizaing with respect to $\w$ and
$b$, and maximizing with respect to $\aVec$. Before taking the derivative of
$L$ with respect to $\w$ and $b$, let's expand $L$
%
\begin{align*}
  \LFunc &= \frac{1}{2}\w^\top\w - \sumN a_n \lb t_n\lp\wTPhiOfXn + b\rp - 1 \rb \\
  \LFunc &= \frac{1}{2}\w^\top\w - \sumN a_nt_n\wTPhiOfXn + a_nt_nb - a_n \\
  \LFunc &= \frac{1}{2}\w^\top\w - \sumN a_nt_n\wTPhiOfXn - \sumN a_nt_nb + \sumN a_n \\
  \LFunc &= \frac{1}{2}\w^\top\w - \sumN a_nt_n\phiOfXnTw - \sumN a_nt_nb + \sumN a_n \\
  \LFunc &= \frac{1}{2}\w^\top\w - \sumN \lb a_nt_n\phiOfXn^\top \rb \w - \sumN \lb a_nt_n \rb b + \sumN a_n \\
\end{align*}
%
Now, taking the derivative of $L$ with respect to $\w$
%
\begin{equation*}
  \dLdw = \w - \sumN a_nt_n\phiOfXn
\end{equation*}
%
Setting this equal to 0 and solving for $\w$
%
\begin{align*}
  0 &= \w - \sumN a_nt_n\phiOfXn \\
  \w &= \sumN a_nt_n\phiOfXn
\end{align*}
%
Taking the derivative of $L$ with respect to $b$
%
\begin{equation*}
  \dLdb = - \sumN a_nt_n
\end{equation*}
%
Setting this equal to 0
%
\begin{equation*}
  0 = \sumN a_nt_n
\end{equation*}
%
\end{document}
